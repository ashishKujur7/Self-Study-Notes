\documentclass[12pt]{article}
\usepackage[margin=1in]{geometry}
\usepackage{amsfonts, amsmath}
\usepackage[T1]{fontenc}
\usepackage{mathrsfs, enumitem}
\usepackage{dirtytalk,hyperref}
\usepackage[utf8]{inputenc}
\usepackage{amssymb}
\usepackage{amsfonts}
\usepackage{amsmath}
\usepackage{amsthm}
\usepackage{color}
\usepackage{hyperref}
\usepackage{csquotes}
\usepackage{fourier}

\newtheorem{theorem}{Theorem}[subsection]
\newtheorem{lemma}[theorem]{Lemma}
\newtheorem{claim}[theorem]{Claim}
\newtheorem{proposition}[theorem]{Proposition}
\newtheorem{corollary}[theorem]{Corollary}
\newtheorem{fact}[theorem]{Fact}
\newtheorem{notation}[theorem]{Notation}
\newtheorem{observation}[theorem]{Observation}
\newtheorem{conjecture}[theorem]{Conjecture}

\theoremstyle{definition}
\newtheorem{definition}[theorem]{Definition}
\newtheorem{example}[theorem]{Example}

\theoremstyle{remark}
\newtheorem{remark}[theorem]{Remark}
\theoremstyle{plain}
\newcommand{\ignore}[1]{}

% section symbol
\renewcommand{\thesection}{\S\arabic{section}}

% \renewcommand{\Pr}{{\bf Pr}}
% \newcommand{\Prx}{\mathop{\bf Pr\/}}
% \newcommand{\E}{{\bf E}}
% \newcommand{\Ex}{\mathop{\bf E\/}}
% \newcommand{\Var}{{\bf Var}}
% \newcommand{\Varx}{\mathop{\bf Var\/}}
% \newcommand{\Cov}{{\bf Cov}}
% \newcommand{\Covx}{\mathop{\bf Cov\/}}

% shortcuts for symbol names that are too long to type
\newcommand{\eps}{\epsilon}
\newcommand{\lam}{\lambda}
\renewcommand{\l}{\ell}
\newcommand{\la}{\langle}
\newcommand{\ra}{\rangle}
\newcommand{\wh}{\widehat}
\newcommand{\wt}{\widetilde}

% % "blackboard-fonted" letters for the reals, naturals etc.
\newcommand{\R}{\mathbb R}
\newcommand{\N}{\mathbb N}
\newcommand{\Z}{\mathbb Z}
\newcommand{\F}{\mathbb F}
\newcommand{\Q}{\mathbb Q}
\newcommand{\C}{\mathbb C}

% % operators that should be typeset in Roman font
% \newcommand{\poly}{\mathrm{poly}}
% \newcommand{\polylog}{\mathrm{polylog}}
% \newcommand{\sgn}{\mathrm{sgn}}
% \newcommand{\avg}{\mathop{\mathrm{avg}}}
% \newcommand{\val}{{\mathrm{val}}}

% % complexity classes
% \renewcommand{\P}{\mathrm{P}}
% \newcommand{\NP}{\mathrm{NP}}
% \newcommand{\BPP}{\mathrm{BPP}}
% \newcommand{\DTIME}{\mathrm{DTIME}}
% \newcommand{\ZPTIME}{\mathrm{ZPTIME}}
% \newcommand{\BPTIME}{\mathrm{BPTIME}}
% \newcommand{\NTIME}{\mathrm{NTIME}}

% values associated to optimization algorithm instances
\newcommand{\Opt}{{\mathsf{Opt}}}
\newcommand{\Alg}{{\mathsf{Alg}}}
\newcommand{\Lp}{{\mathsf{Lp}}}
\newcommand{\Sdp}{{\mathsf{Sdp}}}
\newcommand{\Exp}{{\mathsf{Exp}}}

% if you think the sum and product signs are too big in your math mode; x convention
% as in the probability operators
\newcommand{\littlesum}{{\textstyle \sum}}
\newcommand{\littlesumx}{\mathop{{\textstyle \sum}}}
\newcommand{\littleprod}{{\textstyle \prod}}
\newcommand{\littleprodx}{\mathop{{\textstyle \prod}}}

% horizontal line across the page
\newcommand{\horz}{
\vspace{-.4in}
\begin{center}
\begin{tabular}{p{\textwidth}}\\
\hline
\end{tabular}
\end{center}
}

% calligraphic letters
\newcommand{\calA}{{\cal A}}
\newcommand{\calB}{{\cal B}}
\newcommand{\calC}{{\cal C}}
\newcommand{\calD}{{\cal D}}
\newcommand{\calE}{{\cal E}}
\newcommand{\calF}{{\cal F}}
\newcommand{\calG}{{\cal G}}
\newcommand{\calH}{{\cal H}}
\newcommand{\calI}{{\cal I}}
\newcommand{\calJ}{{\cal J}}
\newcommand{\calK}{{\cal K}}
\newcommand{\calL}{{\cal L}}
\newcommand{\calM}{{\cal M}}
\newcommand{\calN}{{\cal N}}
\newcommand{\calO}{{\cal O}}
\newcommand{\calP}{{\cal P}}
\newcommand{\calQ}{{\cal Q}}
\newcommand{\calR}{{\cal R}}
\newcommand{\calS}{{\cal S}}
\newcommand{\calT}{{\cal T}}
\newcommand{\calU}{{\cal U}}
\newcommand{\calV}{{\cal V}}
\newcommand{\calW}{{\cal W}}
\newcommand{\calX}{{\cal X}}
\newcommand{\calY}{{\cal Y}}
\newcommand{\calZ}{{\cal Z}}

% bold letters (useful for random variables)
%----------------------------------------------
% \renewcommand{\a}{{\boldsymbol a}}
% \renewcommand{\b}{{\boldsymbol b}}
% \renewcommand{\c}{{\boldsymbol c}}
% \renewcommand{\d}{{\boldsymbol d}}
% \newcommand{\e}{{\boldsymbol e}}
% \newcommand{\f}{{\boldsymbol f}}
% \newcommand{\g}{{\boldsymbol g}}
% \newcommand{\h}{{\boldsymbol h}}
% \renewcommand{\i}{{\boldsymbol i}}
% \renewcommand{\j}{{\boldsymbol j}}
% \renewcommand{\k}{{\boldsymbol k}}
% \newcommand{\m}{{\boldsymbol m}}
% \newcommand{\n}{{\boldsymbol n}}
% \renewcommand{\o}{{\boldsymbol o}}
% \newcommand{\p}{{\boldsymbol p}}
% \newcommand{\q}{{\boldsymbol q}}
% \renewcommand{\r}{{\boldsymbol r}}
% \newcommand{\s}{{\boldsymbol s}}
% \renewcommand{\t}{{\boldsymbol t}}
% \renewcommand{\u}{{\boldsymbol u}}
% \renewcommand{\v}{{\boldsymbol v}}
% \newcommand{\w}{{\boldsymbol w}}
% \newcommand{\x}{{\boldsymbol x}}
% \newcommand{\y}{{\boldsymbol y}}
% \newcommand{\z}{{\boldsymbol z}}
% \newcommand{\A}{{\boldsymbol A}}
% \newcommand{\B}{{\boldsymbol B}}
% \newcommand{\C}{{\boldsymbol C}}
% \newcommand{\D}{{\boldsymbol D}}
% \newcommand{\E}{{\boldsymbol E}}
% \newcommand{\F}{{\boldsymbol F}}
% \newcommand{\G}{{\boldsymbol G}}
% \renewcommand{\H}{{\boldsymbol H}}
% \newcommand{\I}{{\boldsymbol I}}
% \newcommand{\J}{{\boldsymbol J}}
% \newcommand{\K}{{\boldsymbol K}}
% \renewcommand{\L}{{\boldsymbol L}}
% \newcommand{\M}{{\boldsymbol M}}
% \renewcommand{\O}{{\boldsymbol O}}
% \renewcommand{\P}{{\mathbb{P}}}
% \newcommand{\Q}{{\boldsymbol Q}}
% \newcommand{\R}{{\boldsymbol R}}
% \renewcommand{\S}{{\boldsymbol S}}
% \newcommand{\T}{{\boldsymbol T}}
% \newcommand{\U}{{\boldsymbol U}}
% \newcommand{\V}{{\boldsymbol V}}
% \newcommand{\W}{{\boldsymbol W}}
% \newcommand{\X}{{\boldsymbol X}}
% \newcommand{\Y}{{\boldsymbol Y}}
% \newcommand{\Z}{{\boldsymbol Z}}

% script letters
\newcommand{\scrA}{{\mathscr A}}
\newcommand{\scrB}{{\mathscr B}}
\newcommand{\scrC}{{\mathscr C}}
\newcommand{\scrD}{{\mathscr D}}
\newcommand{\scrE}{{\mathscr E}}
\newcommand{\scrF}{{\mathscr F}}
\newcommand{\scrG}{{\mathscr G}}
\newcommand{\scrH}{{\mathscr H}}
\newcommand{\scrI}{{\mathscr I}}
\newcommand{\scrJ}{{\mathscr J}}
\newcommand{\scrK}{{\mathscr K}}
\newcommand{\scrL}{{\mathscr L}}
\newcommand{\scrM}{{\mathscr M}}
\newcommand{\scrN}{{\mathscr N}}
\newcommand{\scrO}{{\mathscr O}}
\newcommand{\scrP}{{\mathscr P}}
\newcommand{\scrQ}{{\mathscr Q}}
\newcommand{\scrR}{{\mathscr R}}
\newcommand{\scrS}{{\mathscr S}}
\newcommand{\scrT}{{\mathscr T}}
\newcommand{\scrU}{{\mathscr U}}
\newcommand{\scrV}{{\mathscr V}}
\newcommand{\scrW}{{\mathscr W}}
\newcommand{\scrX}{{\mathscr X}}
\newcommand{\scrY}{{\mathscr Y}}
\newcommand{\scrZ}{{\mathscr Z}}

\title{Chapter 3 --- Convergence in Measure}
\author{}
\date{Last Updated: \today}
\begin{document}

\maketitle
\tableofcontents

\section{Modes of Convergence}

For simplicity, we will deal with $\R$-valued functions only. For further remarks, see the footnote in the book.

\begin{definition}
    Let $\left( X, \scrA , \mu \right)$ be a measure space. Let $f, f_1 ,f_2 ,\ldots : X\to \R$ be a sequence of $\scrA$-measurable functions. We say that $f_n$ converges to $f$ \textit{in measure} if for every $\varepsilon > 0$, we have that
    \begin{equation*}
	\lim_n \mu \left( \left\{ x \in X : \Big| f_n \left( x \right) - f\left( x \right) \Big| > \varepsilon \right\} \right)=0
    \end{equation*}
\end{definition}

\begin{remark}
    \warning \,  General convergence in measure is neither implied nor implies convergence almost everywhere! As the following examples show:
\end{remark}

\begin{example}
    \begin{enumerate}
	\item Consider $\left( X, \scrB (\R) , \lambda \right)$. Consider the sequence of functions $\{\chi _ {[n, \infty)}\}$. Then $\chi_{[n, \infty)} \to 0$ function everywhere (hence, almost everywhere).But it does not converge in measure! To see this take $\varepsilon = \frac{1}{2}$. 
	\item Consider the Borel subsets of $[0, 1)$ with the Lebesgue measure. Consider the sequence whose first term is the characteristic function of $[0,1)$. The next two terms are the characteristic functions of $[0,\frac{1}{2}$ and $[\frac{1}{2}, 1)$. The next four terms are the characteristic functions of $[0, 1/4)$, $[1/4,1/2)$, $[1/2, 3/4)$ and $[3/4, 1)$. Well this goes on\ldots

		Now, let $\varepsilon > 0$ be given. Clearly this sequence converges to zero in measure but for each $x\in [0,1)$, $\left\{ f_n (x) \right\}$ has infinitely many zeroes and ones, so, it does not converge!
    \end{enumerate}
\end{example}

\begin{proposition}[$\mu<\infty$ and $\left\{ f_n \right\}\to f$ a.e. implies $\left\{ f_n \right\}\to f$ in measure]
    Let $(X, \scrA, \mu )$ be a measure space. Let $f, f_1, f_2 \ldots $ be a sequence of $\scrA$-measurable real valued functions on $X$. If $\mu$ is finite and $\left\{ f_n \right\}$ converges almost everywhere to $f$ then $\left\{ f_n \right\}$ converges in measure to $f$.
\end{proposition}
\begin{proof}
   We need to prove that
   $$\lim_n \mu \left( \left\{ x\in X : |f_n (x) -f(x)| > \varepsilon \right\} \right) = 0$$
   for every $\varepsilon > 0$. Let $\varepsilon > 0 $ be given. Define $A_n=\left\{ x\in X : |f_n (x) -f(x)| > \varepsilon \right\}$ and $B_n = \cup _{k\ge n} A_k$.

   It is easy to see that $\left\{ B_n \right\}$ is a decreasing sequence of subsets of $X$. We claim that $\cap_{n} B_n$ is contained in the set $\left\{ x\in X : \left\{ f_n \left( x \right) \right\} \text{does not converge to } f\left( x \right) \right\}$. To see this, let $x\in \cap B_n$, then $x\in A_n$ for infinitely many $n$. If $\left\{ f_n (x) \right\} \to f \left( x \right)$ then there must be some $N\in\N$ such that $|f_n \left( x \right) - f\left( x \right)< \varepsilon$ for every $n\ge N$. Since $x\in A_n$ for infinitely many $n$, we can select a $k\ge N$ such that $x\in A_k$. Then we have that $\varepsilon < |f_k (x) -f(x)| < \varepsilon$ which is absurd! Note that the first inequality is due to $x\in A_k$ and the second inequality holds due to $k \ge N$. Thus, this completes the proof of our claim.

   Notice that the set $\left\{ x\in X : \left\{ f_n (x) \right\} \text{ does not converge to } f\left( x \right) \right\}$ is $\mu$- negligible set. Thus $\mu \left( \cap _n B_n \right) = 0$.

   Since $\left\{ B_n \right\}$ is a decreasing sequence of sets and $\mu < \infty$, Proposition 1.2.5 in the book implies that $\lim _n \mu \left( B_n \right) = \mu \left( \cap _n B_n \right) = 0$.

   Now, observe that $A_n \subseteq B_n$ for every $n\in \N$. This observation implies that $\mu (A_n) \le \mu \left( B_n \right)$ and hence $\mu \left( A_n \right) = 0$ as $n\to \infty$.
   
   This completes the proof.
\end{proof}

\begin{proposition}
    Let $\left( X, \scrA , \mu \right)$ be a measure space. Let $f$ and $f_1 , f_2, \ldots$ be a sequence of $\scrA$ measurable real valued functions on $X$. If $\left\{ f_n \right\}$ converges to $f$ in measure then then there is a subsequence $\left\{ f_{n_k} \right\}$ of $\left\{ f_n \right\}$ which converges almost everywhere to $f$.
\end{proposition}

Let $X$ be a set. A sequence of real valued functions $\left\{ f_n \right\}$ on $X$ is said to \textit{converge uniformly} to a real valued function $f$ on $X$ if for every $\varepsilon > 0$ there is a $N \in \N$ such that for every $n\ge N$ and every $x\in X$, we have that $|f_n (x) - f(x)| < \varepsilon$.

\begin{proposition}[Egoroff's Theoren]
    Let $\left( X, \scrA , \mu \right)$ be a measure space. Let $f$ and $f_1 , f_2 , \ldots$ be a sequence of $\scrA$-measurable real valued functions on $X$. If $\mu$ is finite and $\left\{ f_n \right\}$ converges to $f$ almost everywhere then for each $\varepsilon >0$ there is some $B\in \scrA$ satisfying $\mu \left( B^c \right) < \varepsilon$ and $\left\{ f_n \right\} $ converges to $f$ uniformly on $B$.
\end{proposition}

Egoroff's Theorem provides a motivation for the following definition: 
\begin{definition}
    Let $\left( X, \scrA , \mu \right)$ be a measure space. Let $f$ and $f_1 , f_2 , \ldots$ be a sequence of $\scrA$-measurable real valued functions on $X$. Then $\left\{ f_n \right\}$ converges to $f$ \textit{almost uniformly} if for each $\varepsilon > 0$ there is some $B\in \scrA$ such that $\mu \left( B^c \right) < \varepsilon$ and $\left\{ f_n \right\} $ converges to $f$ uniformly on $B$.
\end{definition}

We remark that if $\left\{ f_n \right\}$ converges to $f$ almost uniformly then $\left\{ f_n \right\}$ converges to $f$ almost everywhere. To see this, for $n\in\N $, select $B_n \in \scrA$ such that $\mu \left( B_n ^c \right) < \frac{1}{n}$ and $\left\{ f_n \right\}$ converges to $f$ uniformly on $B_n$. Let $B=\bigcap_n {B_n}$. Then $\mu (B^c) \le \mu (B_n ^c) < \frac{1}{n} $ for every $n\in\N$. Thus $\mu \left( B^c \right)=0$. We claim that $\lim_n f_n = f$ everywhere on $B$. This is easy to see: if $x\in B$ then $x\in B_n$ for some $n\in\N$ and since $\left\{ f_n \right\}$ converges to $f$ uniformly on $B_n$, we have that $\lim_n f_n \left( x \right) = f\left( x \right)$.

It follows from Egoroff's theorem that on a finite measure space, almost everywhere convergence is equivalent to almost uniform convergence.



\end{document}
