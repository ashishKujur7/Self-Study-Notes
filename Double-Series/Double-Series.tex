\documentclass[12pt]{article}
\usepackage[margin=1in]{geometry}
\usepackage{amsfonts, amsmath}
\usepackage[T1]{fontenc}
\usepackage{mathrsfs, enumitem}
\usepackage{dirtytalk,hyperref}
\usepackage[utf8]{inputenc}
\usepackage{amssymb}
\usepackage{amsfonts}
\usepackage{amsmath}
\usepackage{amsthm}
\usepackage{color}
\usepackage{hyperref}
\usepackage{csquotes}
\usepackage{fourier}

\newtheorem{theorem}{Theorem}[subsection]
\newtheorem{lemma}[theorem]{Lemma}
\newtheorem{claim}[theorem]{Claim}
\newtheorem{proposition}[theorem]{Proposition}
\newtheorem{corollary}[theorem]{Corollary}
\newtheorem{fact}[theorem]{Fact}
\newtheorem{notation}[theorem]{Notation}
\newtheorem{observation}[theorem]{Observation}
\newtheorem{conjecture}[theorem]{Conjecture}

\theoremstyle{definition}
\newtheorem{definition}[theorem]{Definition}
\newtheorem{example}[theorem]{Example}

\theoremstyle{remark}
\newtheorem{remark}[theorem]{Remark}
\theoremstyle{plain}
\newcommand{\ignore}[1]{}

% section symbol
\renewcommand{\thesection}{\S\arabic{section}}

% \renewcommand{\Pr}{{\bf Pr}}
% \newcommand{\Prx}{\mathop{\bf Pr\/}}
% \newcommand{\E}{{\bf E}}
% \newcommand{\Ex}{\mathop{\bf E\/}}
% \newcommand{\Var}{{\bf Var}}
% \newcommand{\Varx}{\mathop{\bf Var\/}}
% \newcommand{\Cov}{{\bf Cov}}
% \newcommand{\Covx}{\mathop{\bf Cov\/}}

% shortcuts for symbol names that are too long to type
\newcommand{\eps}{\epsilon}
\newcommand{\lam}{\lambda}
\renewcommand{\l}{\ell}
\newcommand{\la}{\langle}
\newcommand{\ra}{\rangle}
\newcommand{\wh}{\widehat}
\newcommand{\wt}{\widetilde}

% % "blackboard-fonted" letters for the reals, naturals etc.
\newcommand{\R}{\mathbb R}
\newcommand{\N}{\mathbb N}
\newcommand{\Z}{\mathbb Z}
\newcommand{\F}{\mathbb F}
\newcommand{\Q}{\mathbb Q}
\newcommand{\C}{\mathbb C}

% % operators that should be typeset in Roman font
% \newcommand{\poly}{\mathrm{poly}}
% \newcommand{\polylog}{\mathrm{polylog}}
% \newcommand{\sgn}{\mathrm{sgn}}
% \newcommand{\avg}{\mathop{\mathrm{avg}}}
% \newcommand{\val}{{\mathrm{val}}}

% % complexity classes
% \renewcommand{\P}{\mathrm{P}}
% \newcommand{\NP}{\mathrm{NP}}
% \newcommand{\BPP}{\mathrm{BPP}}
% \newcommand{\DTIME}{\mathrm{DTIME}}
% \newcommand{\ZPTIME}{\mathrm{ZPTIME}}
% \newcommand{\BPTIME}{\mathrm{BPTIME}}
% \newcommand{\NTIME}{\mathrm{NTIME}}

% values associated to optimization algorithm instances
\newcommand{\Opt}{{\mathsf{Opt}}}
\newcommand{\Alg}{{\mathsf{Alg}}}
\newcommand{\Lp}{{\mathsf{Lp}}}
\newcommand{\Sdp}{{\mathsf{Sdp}}}
\newcommand{\Exp}{{\mathsf{Exp}}}

% if you think the sum and product signs are too big in your math mode; x convention
% as in the probability operators
\newcommand{\littlesum}{{\textstyle \sum}}
\newcommand{\littlesumx}{\mathop{{\textstyle \sum}}}
\newcommand{\littleprod}{{\textstyle \prod}}
\newcommand{\littleprodx}{\mathop{{\textstyle \prod}}}

% horizontal line across the page
\newcommand{\horz}{
\vspace{-.4in}
\begin{center}
\begin{tabular}{p{\textwidth}}\\
\hline
\end{tabular}
\end{center}
}

% calligraphic letters
\newcommand{\calA}{{\cal A}}
\newcommand{\calB}{{\cal B}}
\newcommand{\calC}{{\cal C}}
\newcommand{\calD}{{\cal D}}
\newcommand{\calE}{{\cal E}}
\newcommand{\calF}{{\cal F}}
\newcommand{\calG}{{\cal G}}
\newcommand{\calH}{{\cal H}}
\newcommand{\calI}{{\cal I}}
\newcommand{\calJ}{{\cal J}}
\newcommand{\calK}{{\cal K}}
\newcommand{\calL}{{\cal L}}
\newcommand{\calM}{{\cal M}}
\newcommand{\calN}{{\cal N}}
\newcommand{\calO}{{\cal O}}
\newcommand{\calP}{{\cal P}}
\newcommand{\calQ}{{\cal Q}}
\newcommand{\calR}{{\cal R}}
\newcommand{\calS}{{\cal S}}
\newcommand{\calT}{{\cal T}}
\newcommand{\calU}{{\cal U}}
\newcommand{\calV}{{\cal V}}
\newcommand{\calW}{{\cal W}}
\newcommand{\calX}{{\cal X}}
\newcommand{\calY}{{\cal Y}}
\newcommand{\calZ}{{\cal Z}}

% bold letters (useful for random variables)
%----------------------------------------------
% \renewcommand{\a}{{\boldsymbol a}}
% \renewcommand{\b}{{\boldsymbol b}}
% \renewcommand{\c}{{\boldsymbol c}}
% \renewcommand{\d}{{\boldsymbol d}}
% \newcommand{\e}{{\boldsymbol e}}
% \newcommand{\f}{{\boldsymbol f}}
% \newcommand{\g}{{\boldsymbol g}}
% \newcommand{\h}{{\boldsymbol h}}
% \renewcommand{\i}{{\boldsymbol i}}
% \renewcommand{\j}{{\boldsymbol j}}
% \renewcommand{\k}{{\boldsymbol k}}
% \newcommand{\m}{{\boldsymbol m}}
% \newcommand{\n}{{\boldsymbol n}}
% \renewcommand{\o}{{\boldsymbol o}}
% \newcommand{\p}{{\boldsymbol p}}
% \newcommand{\q}{{\boldsymbol q}}
% \renewcommand{\r}{{\boldsymbol r}}
% \newcommand{\s}{{\boldsymbol s}}
% \renewcommand{\t}{{\boldsymbol t}}
% \renewcommand{\u}{{\boldsymbol u}}
% \renewcommand{\v}{{\boldsymbol v}}
% \newcommand{\w}{{\boldsymbol w}}
% \newcommand{\x}{{\boldsymbol x}}
% \newcommand{\y}{{\boldsymbol y}}
% \newcommand{\z}{{\boldsymbol z}}
% \newcommand{\A}{{\boldsymbol A}}
% \newcommand{\B}{{\boldsymbol B}}
% \newcommand{\C}{{\boldsymbol C}}
% \newcommand{\D}{{\boldsymbol D}}
% \newcommand{\E}{{\boldsymbol E}}
% \newcommand{\F}{{\boldsymbol F}}
% \newcommand{\G}{{\boldsymbol G}}
% \renewcommand{\H}{{\boldsymbol H}}
% \newcommand{\I}{{\boldsymbol I}}
% \newcommand{\J}{{\boldsymbol J}}
% \newcommand{\K}{{\boldsymbol K}}
% \renewcommand{\L}{{\boldsymbol L}}
% \newcommand{\M}{{\boldsymbol M}}
% \renewcommand{\O}{{\boldsymbol O}}
% \renewcommand{\P}{{\mathbb{P}}}
% \newcommand{\Q}{{\boldsymbol Q}}
% \newcommand{\R}{{\boldsymbol R}}
% \renewcommand{\S}{{\boldsymbol S}}
% \newcommand{\T}{{\boldsymbol T}}
% \newcommand{\U}{{\boldsymbol U}}
% \newcommand{\V}{{\boldsymbol V}}
% \newcommand{\W}{{\boldsymbol W}}
% \newcommand{\X}{{\boldsymbol X}}
% \newcommand{\Y}{{\boldsymbol Y}}
% \newcommand{\Z}{{\boldsymbol Z}}

% script letters
\newcommand{\scrA}{{\mathscr A}}
\newcommand{\scrB}{{\mathscr B}}
\newcommand{\scrC}{{\mathscr C}}
\newcommand{\scrD}{{\mathscr D}}
\newcommand{\scrE}{{\mathscr E}}
\newcommand{\scrF}{{\mathscr F}}
\newcommand{\scrG}{{\mathscr G}}
\newcommand{\scrH}{{\mathscr H}}
\newcommand{\scrI}{{\mathscr I}}
\newcommand{\scrJ}{{\mathscr J}}
\newcommand{\scrK}{{\mathscr K}}
\newcommand{\scrL}{{\mathscr L}}
\newcommand{\scrM}{{\mathscr M}}
\newcommand{\scrN}{{\mathscr N}}
\newcommand{\scrO}{{\mathscr O}}
\newcommand{\scrP}{{\mathscr P}}
\newcommand{\scrQ}{{\mathscr Q}}
\newcommand{\scrR}{{\mathscr R}}
\newcommand{\scrS}{{\mathscr S}}
\newcommand{\scrT}{{\mathscr T}}
\newcommand{\scrU}{{\mathscr U}}
\newcommand{\scrV}{{\mathscr V}}
\newcommand{\scrW}{{\mathscr W}}
\newcommand{\scrX}{{\mathscr X}}
\newcommand{\scrY}{{\mathscr Y}}
\newcommand{\scrZ}{{\mathscr Z}}

\title{Double Sequences and Series}
\author{Ashish Kujur}
\date{Last Updated : \today}

\begin{document}
\maketitle

\section{Double Sequences}
\subsection{Introduction and Definition}
\begin{definition}
    Let $X$ be a nonempty set. A double sequence $\left( a_{mn} \right)$ is a map $a: \N \times \N \to X$. We write $ a_{m, n} = a\left( m, n \right)$ for all $m,n$. 
    \label{def:doubleSeq}
\end{definition}

In this article, we study the case where $X=\R$ or $X=\C$.

We can view a double sequence $\left( a_{mn} \right)$ as a two dimensional array in the following fashion:

$$\begin{matrix}
    a_{11} & a_{12} & a_{13} & \ldots \\

    a_{21} & a_{22} & a_{23} & \ldots \\
    
    a_{31} & a_{32} & a_{33} & \ldots \\
    
    \hdots & \hdots & \hdots & \ddots
\end{matrix}$$

Suppose that the sequence $\left\{ a_{mn} \right\}_{n\in \N}$ converges to $\alpha _n$ for every $m\in\N$. Furthermore, suppose that the sequence $\left\{ a_{mn} \right\}_{m\in \N}$ converges to $\beta _m$ for every $n\in\N$. Evenmore, let us suppose that $\left\{ \alpha _n \right\} \to \alpha$ and $\left\{ \beta _n \right\} \to \beta$. We may want to define that the series $\left\{ a_{mn} \right\}$ converges if the aforementioned hypothesis hold and $\alpha = \beta$. Definitions are never wrong but this definition is very restrictive as we want too many sequences formed from the double series to be convergent.


After some digression, we come to definition:

\begin{definition}
    Let $\left\{ x_{mn} \right\}_{m,n}$ be a double sequence in $\C$. We say that $\left\{ x_{mn} \right\}_{m,n}$ converges if there is some $l \in \C$ if for every $\varepsilon > 0$ there is some $N \in \N$ such that for every $m,n \ge N$, we have $|x_{m,n} - l| < \varepsilon$.
    \label{def:convDoubleSeq}
\end{definition}

\subsection{Examples of Double Sequences}

\begin{enumerate}
    \item We give an example of a double series whose iterated limits exist and are not equal and the double series is not convergent in the sense of Definition \ref{def:convDoubleSeq}.

	Consider the sequence $\left\{ a_{mn} \right\}$ whose entries are given by $a_{mn}= \begin{cases} 1 & m\ge n \\ 0 & \text{otherwise} \end{cases}$. Here's a two dimensional array representation of the double series $\left\{ a_{mn} \right\}$:

	$$\begin{matrix}
		1 & 0 & 0 & \ldots & 0 & \ldots \\
		1 & 1 & 0 & \ldots & 0 & \ldots \\
		\vdots & \vdots & \vdots & \ddots & \vdots & \vdots & \vdots \\
1 & 1 & 1 & \ldots & 1 & 0 & \ldots  \\
\vdots & \vdots & \vdots & \vdots & \vdots & \vdots
	\end{matrix}$$
	It is rather easy to see that the every row sequence is eventually zero and hence converges to $0$ and that every column sequence is eventually $1$ and hence converges to $1$. But the double sequence is not convergent. If it were convergent, then there must be some $l\in \C$ and $N\in \N$ such that for every $m,n\ge N$, we have that $|a_{mn}-l|\le 1/2$. But note that $|a_{N, N+1} - l| < 1/2$ implies $|l| < 1/2$. But then $|a_{N,N}-l|<1/2$ implies that $|l-1|<1/2$. But note the two aforementioned inequality cannot hold together. 

    \item Consider the double sequence $\left\{ a_{mn} \right\}$ givn by $a_{mn} = \left( -1 \right)^{ m+n} \left( \frac{1}{m} + \frac{1}{n} \right)$. It is very easy to see that the double series $\left\{ a_{m,n} \right\}$ converges to $0$ while none of the row sequences and the column sequences converge!
\end{enumerate}


\subsection{Some Theorem on Double Sequences}

\begin{theorem}
    Let $\left\{ a_{mn} \right\}$ be a double sequence. Suppose that all the row sequences and the column sequences converge and let $\beta _n = \lim _m a_{mn}$ and $\alpha _m = \lim _n a_{mn}$ for every $m,n$ and further suppose that the double sequence converges. Then the following holds
    \begin{equation*}
	\lim_m \alpha_m = \lim_{m,n} a_{mn} = \lim_n \beta_n
    \end{equation*}
    \label{thm:1.1}
\end{theorem}
\begin{proof}
    We first show that $\alpha = \lim_m \alpha _m$ and $\ell = \lim_{m,n} a_{mn}$ are equal. Let $\varepsilon > 0$ be given. Then there is some $N\in \N$ such that for every $n\ge N$, we have that $|a_{mn} - \ell | < \varepsilon$. Taking limit as $n \to \infty$, we get $|\alpha _m - l | \le \varepsilon$ for all $m \ge N$. Taking limit as $m \to \infty$, we have that $| \alpha - \ell | \le \varepsilon $. This proves that $\alpha = \ell$. One can show the other equality similarly.
\end{proof}

\begin{theorem}
    Suppose that a double sequence $\left\{ a_{mn} \right\}$ converges to $\ell$ then the diagonal sequence $\left\{ a_{nn} \right\}_{n}$ converges to $\ell$.
    \label{thm:1.2}
\end{theorem}
\begin{proof}
    Let $\left\{ d_n \right\}$ be the sequence defined by $d_n = a_{nn}$ for every $n$. Let $\varepsilon > 0$ be given. Then there is some $N\in\N$ such that for every $m,n \ge N$, $|a_{mn}- \ell | <\varepsilon$. It is easy to see that for $n\ge N$, we have that $|d_n - \ell| < \varepsilon$. This completes the proof.
\end{proof}

We try to capture Cauchy-ness in this double sequence setting:

\begin{definition}
    Let $\left\{ a_{mn} \right\}$ be a double sequence. We say that $\left\{ a_{mn} \right\}$ is Cauchy if for every $\varepsilon > 0$ there is some $N \in \N$ such that for every $m\ge i \ge N$ and every $n\ge j \ge N$ we have that $|a_{mn}- a_{ij}| < \varepsilon$.
    \label{def:CauchyDoubleSeq}
\end{definition}

As is the case with (one-dimensional) sequences in $\R$, we have the following theorem:

\begin{theorem}
    Every Cauchy double sequence is convergent.
\end{theorem}

\begin{proof}
    Suppose that $\left\{ a_{mn} \right\}$ is a Cauchy double sequence. To show that the double sequence is convergent, we need a candidate for the limit. 
    
    Let $\left\{ d_n \right\}$ be the sequence defined by $d_n = a_{nn}$ for every $n\in\N$. It is easy to see that definition of Cauchy double sequence implies that $\left\{ d_n \right\}$ is Cauchy in $\C$. Thus, it must converge to some $\ell \in \C$.

    We claim that $\left\{ a_{mn} \right\}$ converges to $\ell $. Let $\varepsilon > 0$ be given. Since $\left\{ a_{m,n} \right\}$ is Cauchy, for every $m\ge p \ge N$ and $n \ge p \ge N$, we have that $|a_{mn}-d_p|< \varepsilon /2$.

    Since $\left\{ d_n \right\}$ converges to $\ell$, there is some $M \in \N$ such that $|d_{p} - \ell | < \varepsilon /2$ for every $p\ge M$.
    Let $K=\max \left\{ M,N \right\}$. For every $m,n \ge K$, we have that 
    \begin{align*}
	|a_{mn}-\ell| &< |a_{mn} - d_K| + |d_K - l| \\
    &< \varepsilon /2 + \varepsilon /2 = \varepsilon  	
    \end{align*}
\end{proof}

\end{document}
